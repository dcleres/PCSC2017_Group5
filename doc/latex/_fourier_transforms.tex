Nowadays, the Fourier Transform (FT) is a major tool in all technical domains, raging for image processing to telecommunication. Originally, by computing the FT of a function, the user decomposed a function of time into the frequencies that made it up. The Fourier transform of a function of time is represented in the frequency domain. However, the FT is not limited to functions of time, but in order to have a unified language, the domain of the original function is commonly referred to as the time domain while the domain of the FT is referred to as the frequency domain. For many functions, one can define an operation that reverses this\+: the inverse Fourier transformation of a frequency domain representation combines the contributions of all the different frequencies to recover the original function of time. The fact that one has the possibility to freely oscillate between the time and frequency domain enables the user to compute certain operation which a demanding in the time domain more easily in the frequency domain and then to switch back to the time domain to get the final output. However, in the frame of this project it was possible to take advantage of the fact the Fourier Series are a way to represent a function as a superimposed sum of simple sine waves with different frequencies. More formally, Fourier Series decompose any periodic function into a superimposed sum of complex exponentials. The discrete-\/ time Fourier transform (D\+FT) is a periodic function, often defined in terms of a Fourier series. 